\documentclass{beamer}

\usepackage[czech]{babel}
\usepackage{amsmath}
\usepackage[utf8]{inputenc}
\usepackage{graphicx}
\usepackage{float} 
\usepackage{pgfplots} 
\usepackage{amsfonts}


\graphicspath{ {images/} }
\usetheme{Madrid}


\title{Po částech lineární regrese}
\author[Martin Koběrský]{Martin Koběrský}

\begin{document}
	
	\frame{\titlepage}
	\begin{frame}
	\frametitle{Násobení polynomů}
	\begin{align*}
	p(x) = \sum_{i=0}^{n-1}a_ix^{i}
	\end{align*}
	Součin polynomů můžeme získat takto
	\begin{align*}
	p(x)q(x) = a_0b_0 + (a_0b_1 + a_1b_0)x + (a_0b_2 + a_1b_1 + a_2b_0)x^2 + ... +\\ a_{n-1}b_{n-1}x^{2n-2}.
	\end{align*}
	Stupeň takového polynomu je $n-1$. \\
	Problém s tímto přístupem je, že časová složitost je $O(n^2)$.
\end{frame}

\end{document}
